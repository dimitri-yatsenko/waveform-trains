\documentclass[10pt,twocolumn]{article}
% amsmath package, useful for mathematical formulas
\usepackage{amsmath}
% amssymb package, useful for mathematical symbols
\usepackage{graphicx}
\usepackage{amssymb}
\usepackage{amsmath}
\usepackage{relsize}
\usepackage{fullpage}
%\usepackage{natbib}
\newcommand{\T}{{\sf T}}
\setlength{\parskip}{6pt}
\setlength{\parindent}{0pt}
\newcommand{\E}[1]{\mathbb E\left[ #1\right]}
\newcommand{\R}{\mathbb R}
\newcommand{\N}{\mathbb N}
\newcommand*\diff{\mathop{}\!\mathrm{d}}
\DeclareMathOperator{\var}{var}

\title{Wave train decomposition}
\author{Dimitri Yatsenko}
\begin{document}
\maketitle
This writeup contains an improved algorithm to replace the 2007 version. 

\section*{Formulation}
Let $x(t)\in \R^N$ with $t = 1,\ldots,T$ denote a short (usually 0.5--10 s) epoch of a discretized $N$-channel EEG recording.

We aim to decompose $x(t)$ into a sum of $K$ trains of identical short $N$-channel waveforms $w_k(\tau)\in \R^N$ defined on the interval $-d \le \tau \le +d$:
\begin{equation}
\begin{split}
x(t) = & 
\sum\limits_{k=1}^K \sum_{\tau=-d}^{+d} u_k(t-\tau)w_k(\tau) + \varepsilon(t)
\\
\equiv &
\sum\limits_{k=1}^K \sum_{t^\prime=t-d}^{t+d} u_k(t^\prime)w_k(t-t^\prime) + \varepsilon(t)
\\
\equiv &
\sum\limits_{k=1}^K (u_k*w_k)(t) + \varepsilon(t)
\end{split}
\end{equation}
where $u_k(t)\in \{0,1\}$ is the occurrence indicator function comprising 1s at the time points when an instance of the $k$th waveform occurs in the recording and 0s otherwise; the operator $*$ denotes convolution.

This formulation resembles the \emph{spike sorting} problem for detecting action potentials in multielectrode recordings and assigning them to individual neurons \cite{pillow_model-based_2013}.

\section*{Algorithm}
Until the algorithm converges, the occurrence function is allowed to assume fractional values between 0 and 1.
We define the loss function with a penalty term for the nonbinarity of $u_k(t)$:
\begin{equation}
\mathcal L = 
\sum\limits_{t=1}^T\left(\|\varepsilon(t)\|^2 
+ \lambda \sum\limits_{k=1}^K \left(0.5-u_k(t)\right)^2\right) 
\end{equation}

The gradient of the loss function with respect to the waveform coefficients is 
\begin{equation}
\begin{split}
\frac{\partial \mathcal L}{\partial w_k(\tau)} 
= & -2\sum\limits_{t=1}^T \varepsilon(t) u_k(t-\tau)
\\
\equiv & -2 (\varepsilon\star u_k)(\tau)
\end{split}
\end{equation}
where $\star$ denotes correlation.

Similarly, the gradient with respect to the occurrence indicator function is
\begin{equation}
\begin{split}
\frac{\partial \mathcal L}{\partial u_k(t)} 
= & 
-2 \sum\limits_{t^\prime=1}^T \varepsilon(t^\prime) w_k(t-t^\prime) 
+ 2\lambda(0.5-u_k(t))
\\
= &
-2 (\varepsilon*w_k)(t) 
+ 2\lambda(0.5-u_k(t))
\end{split}
\end{equation}

These gradients provide the update rules for gradient descent: 
\begin{align}
\Delta w_k(\tau) &= \gamma\cdot(\varepsilon \star u_k)(\tau)
\\
\Delta u_k(t) &= \gamma\left[(\varepsilon * w_k)(t)
- \lambda (1-2u_k(t))\right]
\end{align}
with bounded $u_k(t)\in [0,1]$.
The update coefficients $\gamma$ decreases with the convergence of the algorithm while the nonbinarity penalty coefficient $\lambda$  increases.

\bibliographystyle{plain}
\bibliography{SpikeSorting}

\end{document}
